\documentclass[a4paper, 11pt]{book}

%\usepackage[center]{titlesec}

\usepackage{amsfonts, amssymb, amsmath, amsthm, amsxtra}

\usepackage{foekfont}

\usepackage{MnSymbol}

\usepackage{pdfrender, xcolor}
%\pdfrender{StrokeColor=black,LineWidth=.4pt,TextRenderingMode=2}

\usepackage{minitoc}
\setcounter{tocdepth}{5}
\setcounter{minitocdepth}{5}
\setcounter{secnumdepth}{5}

\usepackage{graphicx}

\usepackage[english]{babel}
\usepackage[utf8]{inputenc}
%\usepackage{mathpazo}
%\usepackage{euler}
\usepackage{eucal}
\usepackage{bbm}
\usepackage{bm}
\usepackage{csquotes}
\usepackage[nottoc]{tocbibind}
\usepackage{appendix}
\usepackage{float}
\usepackage[T1]{fontenc}
\usepackage[
    left = \flqq{},% 
    right = \frqq{},% 
    leftsub = \flq{},% 
    rightsub = \frq{} %
]{dirtytalk}

\usepackage{imakeidx}
\makeindex

%\usepackage[dvipsnames]{xcolor}
\usepackage{hyperref}
    \hypersetup{
        colorlinks=true,
        linkcolor=teal,
        filecolor=pink,      
        urlcolor=teal,
        citecolor=magenta
    }
\usepackage{comment}

% You would set the PDF title, author, etc. with package options or
% \hypersetup.

\usepackage[backend=biber, style=alphabetic, sorting=nty]{biblatex}
    \addbibresource{bibliography.bib}

\raggedbottom

\usepackage{mathrsfs}
\usepackage{mathtools} 
\mathtoolsset{showonlyrefs} 
%\usepackage{amssymb}
%\usepackage{amsthm}
\renewcommand\qedsymbol{$\blacksquare$}
\usepackage{tikz-cd}
\tikzcdset{scale cd/.style={every label/.append style={scale=#1},
    cells={nodes={scale=#1}}}}
\usepackage{tikz}
\usepackage{setspace}
\usepackage[version=3]{mhchem}
\parskip=0.1in
\usepackage[margin=25mm]{geometry}

\usepackage{listings, lstautogobble}
\lstset{
	language=matlab,
	basicstyle=\scriptsize\ttfamily,
	commentstyle=\ttfamily\itshape\color{gray},
	stringstyle=\ttfamily,
	showstringspaces=false,
	breaklines=true,
	frameround=ffff,
	frame=single,
	rulecolor=\color{black},
	autogobble=true
}

\usepackage{todonotes,tocloft,xpatch,hyperref}

% This is based on classicthesis chapter definition
\let\oldsec=\section
\renewcommand*{\section}{\secdef{\Sec}{\SecS}}
\newcommand\SecS[1]{\oldsec*{#1}}%
\newcommand\Sec[2][]{\oldsec[\texorpdfstring{#1}{#1}]{#2}}%

\newcounter{istodo}[section]

% http://tex.stackexchange.com/a/61267/11984
\makeatletter
%\xapptocmd{\Sec}{\addtocontents{tdo}{\protect\todoline{\thesection}{#1}{}}}{}{}
\newcommand{\todoline}[1]{\@ifnextchar\Endoftdo{}{\@todoline{#1}}}
\newcommand{\@todoline}[3]{%
	\@ifnextchar\todoline{}
	{\contentsline{section}{\numberline{#1}#2}{#3}{}{}}%
}
\let\l@todo\l@subsection
\newcommand{\Endoftdo}{}

\AtEndDocument{\addtocontents{tdo}{\string\Endoftdo}}
\makeatother

\usepackage{lipsum}

%   Reduce the margin of the summary:
\def\changemargin#1#2{\list{}{\rightmargin#2\leftmargin#1}\item[]}
\let\endchangemargin=\endlist 

%   Generate the environment for the abstract:
\newcommand\summaryname{Abstract}
\newenvironment{abstract}%
    {\small\begin{center}%
    \bfseries{\summaryname} \end{center}}

\newtheorem{theorem}{Theorem}[section]
    \numberwithin{theorem}{subsection}
\newtheorem{proposition}{Proposition}[section]
    \numberwithin{proposition}{subsection}
\newtheorem{lemma}{Lemma}[section]
    \numberwithin{lemma}{subsection}
\newtheorem{claim}{Claim}[section]
    \numberwithin{claim}{subsection}

\theoremstyle{definition}
    \newtheorem{definition}{Definition}[section]
        \numberwithin{definition}{subsection}

\theoremstyle{remark}
    \newtheorem{remark}{Remark}[section]
        \numberwithin{remark}{subsection}
    \newtheorem{example}{Example}[section]
        \numberwithin{example}{subsection}    
    \newtheorem{convention}{Convention}[section]
        \numberwithin{convention}{subsection}
    \newtheorem{corollary}{Corollary}[section]
        \numberwithin{corollary}{subsection}

\setcounter{section}{-1}

\renewcommand{\cong}{\simeq}
\newcommand{\ladjoint}{\dashv}
\newcommand{\radjoint}{\vdash}
\newcommand{\<}{\langle}
\renewcommand{\>}{\rangle}
\newcommand{\ndiv}{\hspace{-2pt}\not|\hspace{5pt}}
\newcommand{\cond}{\blacktriangle}
\newcommand{\decond}{\triangle}
\newcommand{\solid}{\blacksquare}
\newcommand{\ot}{\leftarrow}
\renewcommand{\-}{\text{-}}
\renewcommand{\mapsto}{\leadsto}
\renewcommand{\leq}{\leqslant}
\renewcommand{\geq}{\geqslant}
\renewcommand{\setminus}{\smallsetminus}
\makeatletter
\DeclareRobustCommand{\cev}[1]{%
  {\mathpalette\do@cev{#1}}%
}
\newcommand{\do@cev}[2]{%
  \vbox{\offinterlineskip
    \sbox\z@{$\m@th#1 x$}%
    \ialign{##\cr
      \hidewidth\reflectbox{$\m@th#1\vec{}\mkern4mu$}\hidewidth\cr
      \noalign{\kern-\ht\z@}
      $\m@th#1#2$\cr
    }%
  }%
}
\makeatother

\newcommand{\N}{\mathbb{N}}
\newcommand{\Z}{\mathbb{Z}}
\newcommand{\Q}{\mathbb{Q}}
\newcommand{\R}{\mathbb{R}}
\newcommand{\bbC}{\mathbb{C}}
\NewDocumentCommand{\x}{e{_^}}{%
  \mathbin{\mathop{\times}\displaylimits
    \IfValueT{#1}{_{#1}}
    \IfValueT{#2}{^{#2}}
  }%
}
\NewDocumentCommand{\pushout}{e{_^}}{%
  \mathbin{\mathop{\sqcup}\displaylimits
    \IfValueT{#1}{_{#1}}
    \IfValueT{#2}{^{#2}}
  }%
}
\newcommand{\supp}{\operatorname{supp}}
\newcommand{\im}{\operatorname{im}}
\newcommand{\coker}{\operatorname{coker}}
\newcommand{\id}{\mathrm{id}}
\newcommand{\chara}{\operatorname{char}}
\newcommand{\trdeg}{\operatorname{trdeg}}
\newcommand{\rank}{\operatorname{rank}}
\newcommand{\trace}{\operatorname{tr}}
\newcommand{\length}{\operatorname{length}}
\newcommand{\height}{\operatorname{height}}
\renewcommand{\span}{\operatorname{span}}
\newcommand{\e}{\epsilon}
\newcommand{\p}{\mathfrak{p}}
\newcommand{\q}{\mathfrak{q}}
\newcommand{\m}{\mathfrak{m}}
\newcommand{\n}{\mathfrak{n}}
\newcommand{\calF}{\mathcal{F}}
\newcommand{\calG}{\mathcal{G}}
\newcommand{\calO}{\mathcal{O}}
\newcommand{\F}{\mathbb{F}}
\DeclareMathOperator{\lcm}{lcm}
\newcommand{\gr}{\operatorname{gr}}
\newcommand{\vol}{\mathrm{vol}}
\newcommand{\ord}{\operatorname{ord}}

\newcommand{\GL}{\operatorname{GL}}
\newcommand{\SL}{\operatorname{SL}}
\newcommand{\Sp}{\operatorname{Sp}}
\newcommand{\GSp}{\operatorname{GSp}}
\newcommand{\GSpin}{\operatorname{GSpin}}
\newcommand{\opO}{\operatorname{O}}
\newcommand{\SO}{\operatorname{SO}}
\newcommand{\SU}{\operatorname{SU}}
\newcommand{\opU}{\operatorname{U}}
\newcommand{\Spec}{\operatorname{Spec}}
\newcommand{\Spf}{\operatorname{Spf}}
\newcommand{\Spm}{\operatorname{Spm}}
\newcommand{\Spv}{\operatorname{Spv}}
\newcommand{\Spa}{\operatorname{Spa}}
\newcommand{\Spd}{\operatorname{Spd}}
\newcommand{\Proj}{\operatorname{Proj}}
\newcommand{\Gr}{\mathrm{Gr}}
\newcommand{\Hecke}{\mathrm{Hecke}}
\newcommand{\Sht}{\mathrm{Sht}}
\newcommand{\Quot}{\mathrm{Quot}}
\newcommand{\Hilb}{\mathrm{Hilb}}
\newcommand{\Pic}{\mathrm{Pic}}
\newcommand{\Div}{\mathrm{Div}}
\newcommand{\Jac}{\mathrm{Jac}}
\newcommand{\Alb}{\mathrm{Alb}} %albanese variety
\newcommand{\Bun}{\mathrm{Bun}}
\newcommand{\loopspace}{\mathbf{\Omega}}
\newcommand{\suspension}{\mathbf{\Sigma}}
\newcommand{\tangent}{\mathrm{T}} %tangent space
\newcommand{\Eig}{\mathrm{Eig}}

\newcommand{\Ring}{\mathrm{Ring}}
\newcommand{\Cring}{\mathrm{CRing}}
\newcommand{\Alg}{\mathrm{Alg}}
\newcommand{\Leib}{\mathrm{Leib}} %leibniz algebras
\newcommand{\Fld}{\mathrm{Fld}}
\newcommand{\Sets}{\mathrm{Sets}}
\newcommand{\Cat}{\mathrm{Cat}}
\newcommand{\Grp}{\mathrm{Grp}}
\newcommand{\Ab}{\mathrm{Ab}}
\newcommand{\Sch}{\mathrm{Sch}}
\newcommand{\Coh}{\mathrm{Coh}}
\newcommand{\QCoh}{\mathrm{QCoh}}
\newcommand{\Perf}{\mathrm{Perf}} %perfect complexes
\newcommand{\Sing}{\mathrm{Sing}} %singularity categories
\newcommand{\Desc}{\mathrm{Desc}}
\newcommand{\Sh}{\mathrm{Sh}}
\newcommand{\Psh}{\mathrm{PSh}}
\newcommand{\Fib}{\mathrm{Fib}}
\renewcommand{\mod}{\-\mathrm{mod}}
\newcommand{\comod}{\-\mathrm{comod}}
\newcommand{\bimod}{\-\mathrm{bimod}}
\newcommand{\Vect}{\mathrm{Vect}}
\newcommand{\Rep}{\mathrm{Rep}}
\newcommand{\Grpd}{\mathrm{Grpd}}
\newcommand{\Arr}{\mathrm{Arr}}
\newcommand{\Esp}{\mathrm{Esp}}
\newcommand{\Ob}{\mathrm{Ob}}
\newcommand{\Mor}{\mathrm{Mor}}
\newcommand{\Man}{\mathrm{Man}}
\newcommand{\Riem}{\mathrm{Riem}}
\newcommand{\RS}{\mathrm{RS}}
\newcommand{\LRS}{\mathrm{LRS}}
\newcommand{\TRS}{\mathrm{TRS}}
\newcommand{\LVRS}{\mathrm{LVRS}}
\newcommand{\Spc}{\mathrm{Spc}}
\newcommand{\Top}{\mathrm{Top}}
\newcommand{\Topos}{\mathrm{Topos}}
\newcommand{\Nil}{\mathfrak{Nil}}
\newcommand{\J}{\mathfrak{J}}
\newcommand{\Stk}{\mathrm{Stk}}
\newcommand{\Pre}{\mathrm{Pre}}
\newcommand{\simp}{\mathbf{\Delta}}
\newcommand{\Res}{\mathrm{Res}}
\newcommand{\Ind}{\mathrm{Ind}}
\newcommand{\Pro}{\mathrm{Pro}}
\newcommand{\Mon}{\mathrm{Mon}}
\newcommand{\Comm}{\mathrm{Comm}}
\newcommand{\Fin}{\mathrm{Fin}}
\newcommand{\Assoc}{\mathrm{Assoc}}
\newcommand{\Semi}{\mathrm{Semi}}
\newcommand{\Co}{\mathrm{Co}}
\newcommand{\Loc}{\mathrm{Loc}}
\newcommand{\Ringed}{\mathrm{Ringed}}
\newcommand{\Haus}{\mathrm{Haus}} %hausdorff spaces
\newcommand{\Comp}{\mathrm{Comp}} %compact hausdorff spaces
\newcommand{\Stone}{\mathrm{Stone}} %stone spaces
\newcommand{\Extr}{\mathrm{Extr}} %extremely disconnected spaces
\newcommand{\Ouv}{\mathrm{Ouv}}
\newcommand{\Str}{\mathrm{Str}}
\newcommand{\Func}{\mathrm{Func}}
\newcommand{\Crys}{\mathrm{Crys}}
\newcommand{\LocSys}{\mathrm{LocSys}}
\newcommand{\Sieves}{\mathrm{Sieves}}
\newcommand{\pt}{\mathrm{pt}}
\newcommand{\Graphs}{\mathrm{Graphs}}
\newcommand{\Lie}{\mathrm{Lie}}
\newcommand{\Env}{\mathrm{Env}}
\newcommand{\Ho}{\mathrm{Ho}}
\newcommand{\rmD}{\mathrm{D}}
\newcommand{\Cov}{\mathrm{Cov}}
\newcommand{\Frames}{\mathrm{Frames}}
\newcommand{\Locales}{\mathrm{Locales}}
\newcommand{\Span}{\mathrm{Span}}
\newcommand{\Corr}{\mathrm{Corr}}
\newcommand{\Monad}{\mathrm{Monad}}
\newcommand{\Var}{\mathrm{Var}}
\newcommand{\sfN}{\mathrm{N}} %nerve
\newcommand{\Dia}{\mathrm{Dia}}
\newcommand{\co}{\mathrm{co}}
\newcommand{\ev}{\mathrm{ev}}
\newcommand{\bi}{\mathrm{bi}}
\newcommand{\Nat}{\mathrm{Nat}}
\newcommand{\Hopf}{\mathrm{Hopf}}
\newcommand{\Dmod}{\mathrm{D}\mod}
\newcommand{\Perv}{\mathrm{Perv}}
\newcommand{\Sph}{\mathrm{Sph}}
\newcommand{\Moduli}{\mathrm{Moduli}}
\newcommand{\Pseudo}{\mathrm{Pseudo}}
\newcommand{\Lax}{\mathrm{Lax}}
\newcommand{\Strict}{\mathrm{Strict}}
\newcommand{\Opd}{\mathrm{Opd}} %operads
\newcommand{\Shv}{\mathrm{Shv}}
\newcommand{\Char}{\mathrm{Char}} %CharShv = character sheaves
\newcommand{\Huber}{\mathrm{Huber}}
\newcommand{\Tate}{\mathrm{Tate}}
\newcommand{\Affd}{\mathrm{Affd}} %affinoid algebras
\newcommand{\Adic}{\mathrm{Adic}} %adic spaces
\newcommand{\Rig}{\mathrm{Rig}}
\newcommand{\An}{\mathrm{An}}
\newcommand{\Perfd}{\mathrm{Perfd}} %perfectoid spaces
\newcommand{\Sub}{\mathrm{Sub}} %subobjects
\newcommand{\Ideals}{\mathrm{Ideals}}
\newcommand{\Isoc}{\mathrm{Isoc}} %isocrystals
\newcommand{\Ban}{\-\mathrm{Ban}} %Banach spaces
\newcommand{\Fre}{\-\mathrm{Fre}} %Frechet spaces
\newcommand{\Ch}{\mathrm{Ch}} %chain complexes
\newcommand{\Pure}{\mathrm{Pure}}
\newcommand{\Mixed}{\mathrm{Mixed}}
\newcommand{\Hodge}{\mathrm{Hodge}} %Hodge structures
\newcommand{\Mot}{\mathrm{Mot}} %motives
\newcommand{\KL}{\mathrm{KL}} %category of Kazhdan-Lusztig modules
\newcommand{\Pres}{\mathrm{Pres}} %presentable categories
\newcommand{\Noohi}{\mathrm{Noohi}} %category of Noohi groups
\newcommand{\Inf}{\mathrm{Inf}}
\newcommand{\LPar}{\mathrm{LPar}} %Langlands parameters
\newcommand{\ORig}{\mathrm{ORig}} %overconvergent sites

\newcommand{\Aut}{\mathrm{Aut}}
\newcommand{\Inn}{\mathrm{Inn}}
\newcommand{\Out}{\mathrm{Out}}
\newcommand{\gl}{\mathfrak{gl}}
\renewcommand{\sl}{\mathfrak{sl}}
\newcommand{\der}{\mathfrak{der}} %derivations on Lie algebras
\newcommand{\inn}{\mathfrak{inn}} %inner derivations
\newcommand{\out}{\mathfrak{out}} %outer derivations
\newcommand{\Stab}{\mathrm{Stab}}
\newcommand{\Cent}{\mathrm{Cent}}
\newcommand{\Norm}{\mathrm{Norm}}
\newcommand{\Rad}{\mathrm{Rad}}
\newcommand{\Transporter}{\mathrm{Transp}} %transporter between two subsets of a group
\newcommand{\Conj}{\mathrm{Conj}}
\newcommand{\Diag}{\mathrm{Diag}}
\newcommand{\Gal}{\mathrm{Gal}}
\newcommand{\bfG}{\mathbf{G}} %absolute Galois group
\newcommand{\Frac}{\mathrm{Frac}}
\newcommand{\Ann}{\mathrm{Ann}}
\newcommand{\Val}{\mathrm{Val}}
\newcommand{\Chow}{\mathrm{Chow}}
\newcommand{\Sym}{\mathrm{Sym}}
\newcommand{\End}{\mathrm{End}}
\newcommand{\Mat}{\mathrm{Mat}}
\newcommand{\Diff}{\mathrm{Diff}}
\newcommand{\Autom}{\mathrm{Autom}}
\newcommand{\Artin}{\mathrm{Artin}} %artin maps
\newcommand{\sk}{\mathrm{sk}} %skeleton of a category
\newcommand{\eqv}{\mathrm{eqv}} %functor that maps groups $G$ to $G$-sets
\newcommand{\Inert}{\mathrm{Inert}}
\newcommand{\Fil}{\mathrm{Fil}}
\newcommand{\Prim}{\mathfrak{Prim}}
\newcommand{\Nerve}{\mathrm{N}}
\newcommand{\Hol}{\mathrm{Hol}} %holomorphic functions %holonomy groups
\newcommand{\Bi}{\mathrm{Bi}} %Bi for biholomorphic functions

\newcommand{\colim}{\operatorname{colim} \:}
\renewcommand{\lim}{\operatorname{lim} \:}
\newcommand{\toto}{\rightrightarrows}
%\newcommand{\tensor}{\otimes}
\NewDocumentCommand{\tensor}{e{_^}}{%
  \mathbin{\mathop{\otimes}\displaylimits
    \IfValueT{#1}{_{#1}}
    \IfValueT{#2}{^{#2}}
  }%
}
\NewDocumentCommand{\hattensor}{e{_^}}{%
  \mathbin{\mathop{\hat{\otimes}}\displaylimits
    \IfValueT{#1}{_{#1}}
    \IfValueT{#2}{^{#2}}
  }%
}
\NewDocumentCommand{\semidirect}{e{_^}}{%
  \mathbin{\mathop{\rtimes}\displaylimits
    \IfValueT{#1}{_{#1}}
    \IfValueT{#2}{^{#2}}
  }%
}
\newcommand{\eq}{\operatorname{eq}}
\newcommand{\coeq}{\operatorname{coeq}}
\newcommand{\Hom}{\mathrm{Hom}}
\newcommand{\Maps}{\mathrm{Maps}}
\newcommand{\Tor}{\mathrm{Tor}}
\newcommand{\Ext}{\mathrm{Ext}}
\newcommand{\Isom}{\mathrm{Isom}}
\newcommand{\stalk}{\mathrm{stalk}}
\newcommand{\RKE}{\operatorname{RKE}}
\newcommand{\LKE}{\operatorname{LKE}}
\newcommand{\oblv}{\mathrm{oblv}}
\newcommand{\const}{\mathrm{const}}
%\newcommand{\forget}{\mathrm{forget}}
\newcommand{\adrep}{\mathrm{ad}} %adjoint representation
\newcommand{\NL}{\mathbb{NL}} %naive cotangent complex
\newcommand{\pr}{\operatorname{pr}}
\newcommand{\Der}{\mathrm{Der}}
\newcommand{\Frob}{\mathrm{Fr}} %Frobenius
\newcommand{\frob}{\mathrm{f}} %trace of Frobenius
\newcommand{\bfpt}{\mathbf{pt}}
\newcommand{\bfloc}{\mathbf{loc}}
\DeclareMathAlphabet{\mymathbb}{U}{BOONDOX-ds}{m}{n}
\newcommand{\0}{\mymathbb{0}}
\newcommand{\1}{\mathbbm{1}}
\newcommand{\2}{\mathbbm{2}}
\newcommand{\Jet}{\mathrm{Jet}}
\newcommand{\Split}{\mathrm{Split}}
\newcommand{\Sq}{\mathrm{Sq}}
\newcommand{\Zero}{\mathrm{Z}}
\newcommand{\SqZ}{\Sq\Zero}
\newcommand{\frakLie}{\mathfrak{Lie}}
\newcommand{\y}{\mathrm{y}} %yoneda
\newcommand{\Sm}{\mathrm{Sm}}
\newcommand{\AJ}{\phi} %abel-jacobi map
\newcommand{\act}{\mathrm{act}}
\newcommand{\ram}{\mathrm{ram}} %ramification index
\newcommand{\inv}{\mathrm{inv}}
\newcommand{\Spr}{\mathrm{Spr}} %the Springer map/sheaf
\newcommand{\Def}{\mathrm{Def}}

\newcommand{\bbU}{\mathbb{U}}
\newcommand{\V}{\mathbb{V}}
\newcommand{\U}{\mathrm{U}}
\newcommand{\calU}{\mathcal{U}}
\newcommand{\calW}{\mathcal{W}}
\newcommand{\rmI}{\mathrm{I}} %augmentation ideal
\newcommand{\bfV}{\mathbf{V}}
\newcommand{\C}{\mathcal{C}}
\newcommand{\D}{\mathcal{D}}
\newcommand{\T}{\mathscr{T}} %Tate modules
\newcommand{\calM}{\mathcal{M}}
\newcommand{\calN}{\mathcal{N}}
\newcommand{\calP}{\mathcal{P}}
\newcommand{\calQ}{\mathcal{Q}}
\newcommand{\A}{\mathbb{A}}
\renewcommand{\P}{\mathbb{P}}
\newcommand{\calL}{\mathcal{L}}
\newcommand{\E}{\mathcal{E}}
\renewcommand{\H}{\mathbf{H}}
\newcommand{\scrS}{\mathscr{S}}
\newcommand{\calX}{\mathcal{X}}
\newcommand{\calY}{\mathcal{Y}}
\newcommand{\calZ}{\mathcal{Z}}
\newcommand{\calS}{\mathcal{S}}
\newcommand{\calR}{\mathcal{R}}
\newcommand{\scrX}{\mathscr{X}}
\newcommand{\scrY}{\mathscr{Y}}
\newcommand{\scrZ}{\mathscr{Z}}
\newcommand{\calA}{\mathcal{A}}
\newcommand{\calB}{\mathcal{B}}
\renewcommand{\S}{\mathcal{S}}
\newcommand{\B}{\mathbb{B}}
\newcommand{\bbD}{\mathbb{D}}
\newcommand{\G}{\mathbb{G}}
\newcommand{\horn}{\mathbf{\Lambda}}
\renewcommand{\L}{\mathbb{L}}
\renewcommand{\a}{\mathfrak{a}}
\renewcommand{\b}{\mathfrak{b}}
\renewcommand{\t}{\mathfrak{t}}
\renewcommand{\r}{\mathfrak{r}}
\newcommand{\bbX}{\mathbb{X}}
\newcommand{\frakG}{\mathfrak{G}}
\newcommand{\frakH}{\mathfrak{H}}
\newcommand{\g}{\mathfrak{g}}
\newcommand{\h}{\mathfrak{h}}
\renewcommand{\k}{\mathfrak{k}}
\newcommand{\del}{\partial}
\newcommand{\bbE}{\mathbb{E}}
\newcommand{\scrO}{\mathscr{O}}
\newcommand{\bbO}{\mathbb{O}}
\newcommand{\scrA}{\mathscr{A}}
\newcommand{\scrB}{\mathscr{B}}
\newcommand{\scrF}{\mathscr{F}}
\newcommand{\scrG}{\mathscr{G}}
\newcommand{\scrM}{\mathscr{M}}
\newcommand{\scrN}{\mathscr{N}}
\newcommand{\scrP}{\mathscr{P}}
\newcommand{\frakS}{\mathfrak{S}}
\newcommand{\frakT}{\mathfrak{T}}
\newcommand{\calI}{\mathcal{I}}
\newcommand{\calJ}{\mathcal{J}}
\newcommand{\scrI}{\mathscr{I}}
\newcommand{\scrJ}{\mathscr{J}}
\newcommand{\scrK}{\mathscr{K}}
\newcommand{\calK}{\mathcal{K}}
\newcommand{\scrV}{\mathscr{V}}
\newcommand{\scrW}{\mathscr{W}}
\newcommand{\bbS}{\mathbb{S}}
\newcommand{\scrH}{\mathscr{H}}
\newcommand{\bfA}{\mathbf{A}}
\newcommand{\bfB}{\mathbf{B}}
\newcommand{\bfC}{\mathbf{C}}
\newcommand{\Witt}{W}
\renewcommand{\O}{\mathbb{O}}
\newcommand{\calV}{\mathcal{V}}
\newcommand{\scrR}{\mathscr{R}} %radical
\newcommand{\rmZ}{\mathrm{Z}} %centre of algebra
\newcommand{\bfGamma}{\mathbf{\Gamma}}
\newcommand{\scrU}{\mathscr{U}}
\newcommand{\rmW}{\mathrm{W}} %Weil group
\newcommand{\frakM}{\mathfrak{M}}
\newcommand{\frakN}{\mathfrak{N}}
\newcommand{\frakX}{\mathfrak{X}}
\newcommand{\frakY}{\mathfrak{Y}}
\newcommand{\frakZ}{\mathfrak{Z}}
\newcommand{\frakU}{\mathfrak{U}}
\newcommand{\frakR}{\mathfrak{R}}
\newcommand{\rmC}{\mathrm{C}}
\newcommand{\frakP}{\mathfrak{P}}
\newcommand{\frakQ}{\mathfrak{Q}}
\newcommand{\frakE}{\mathfrak{E}}
\newcommand{\sfX}{\mathsf{X}}
\newcommand{\sfY}{\mathsf{Y}}
\newcommand{\sfZ}{\mathsf{Z}}
\newcommand{\sfS}{\mathsf{S}}
\newcommand{\sfT}{\mathsf{T}}
\newcommand{\sfOmega}{\mathsf{\Omega}} %drinfeld p-adic upper-half plane

\newcommand{\aff}{\mathrm{aff}}
\newcommand{\ft}{\mathrm{ft}} %finite type
\newcommand{\fp}{\mathrm{fp}} %finite presentation
\newcommand{\aft}{\mathrm{aft}}
\newcommand{\lft}{\mathrm{lft}}
\newcommand{\laft}{\mathrm{laft}}
\newcommand{\cpt}{\mathrm{cpt}}
\newcommand{\cproj}{\mathrm{cproj}}
\newcommand{\qc}{\mathrm{qc}}
\newcommand{\qs}{\mathrm{qs}}
\newcommand{\lcmpt}{\mathrm{lcmpt}}
\newcommand{\red}{\mathrm{red}}
\newcommand{\fin}{\mathrm{fin}}
\newcommand{\gen}{\mathrm{gen}}
\newcommand{\petit}{\mathrm{petit}}
\newcommand{\gros}{\mathrm{gros}}
\newcommand{\loc}{\mathrm{loc}}
\newcommand{\glob}{\mathrm{glob}}
%\newcommand{\ringed}{\mathrm{ringed}}
%\newcommand{\qcoh}{\mathrm{qcoh}}
\newcommand{\cl}{\mathrm{cl}}
\newcommand{\et}{\mathrm{\acute{e}t}}
\newcommand{\fet}{\mathrm{f\acute{e}t}}
\newcommand{\profet}{\mathrm{prof\acute{e}t}}
\newcommand{\proet}{\mathrm{pro\acute{e}t}}
\newcommand{\Zar}{\mathrm{Zar}}
\newcommand{\fppf}{\mathrm{fppf}}
\newcommand{\fpqc}{\mathrm{fpqc}}
\newcommand{\orig}{\mathrm{orig}} %overconvergent topology
\newcommand{\smooth}{\mathrm{sm}}
\newcommand{\sh}{\mathrm{sh}}
\newcommand{\op}{\mathrm{op}}
\newcommand{\open}{\mathrm{open}}
\newcommand{\closed}{\mathrm{closed}}
\newcommand{\geom}{\mathrm{geom}}
\newcommand{\alg}{\mathrm{alg}}
\newcommand{\sober}{\mathrm{sober}}
\newcommand{\dR}{\mathrm{dR}}
\newcommand{\rad}{\mathfrak{Rad}}
\newcommand{\discrete}{\mathrm{discrete}}
%\newcommand{\add}{\mathrm{add}}
%\newcommand{\lin}{\mathrm{lin}}
\newcommand{\Krull}{\mathrm{Krull}}
\newcommand{\qis}{\mathrm{qis}} %quasi-isomorphism
\newcommand{\ho}{\mathrm{ho}} %homotopy equivalence
\newcommand{\sep}{\mathrm{sep}}
\newcommand{\unr}{\mathrm{unr}}
\newcommand{\tame}{\mathrm{tame}}
\newcommand{\wild}{\mathrm{wild}}
\newcommand{\nil}{\mathrm{nil}}
\newcommand{\defm}{\mathrm{defm}}
\newcommand{\Art}{\mathrm{Art}}
\newcommand{\Noeth}{\mathrm{Noeth}}
\newcommand{\affd}{\mathrm{affd}}
%\newcommand{\adic}{\mathrm{adic}}
\newcommand{\pre}{\mathrm{pre}}
\newcommand{\coperf}{\mathrm{coperf}}
\newcommand{\perf}{\mathrm{perf}}
\newcommand{\perfd}{\mathrm{perfd}}
\newcommand{\rat}{\mathrm{rat}}
\newcommand{\cont}{\mathrm{cont}}
\newcommand{\dg}{\mathrm{dg}}
\newcommand{\almost}{\mathrm{a}}
%\newcommand{\stab}{\mathrm{stab}}
\newcommand{\heart}{\heartsuit}
\newcommand{\proj}{\mathrm{proj}}
\newcommand{\qproj}{\mathrm{qproj}}
\newcommand{\pd}{\mathrm{pd}}
\newcommand{\crys}{\mathrm{crys}}
\newcommand{\prisma}{\mathrm{prisma}}
\newcommand{\FF}{\mathrm{FF}}
\newcommand{\sph}{\mathrm{sph}}
\newcommand{\lax}{\mathrm{lax}}
\newcommand{\weak}{\mathrm{weak}}
\newcommand{\strict}{\mathrm{strict}}
\newcommand{\mon}{\mathrm{mon}}
\newcommand{\sym}{\mathrm{sym}}
\newcommand{\lisse}{\mathrm{lisse}}
\newcommand{\an}{\mathrm{an}}
\newcommand{\ad}{\mathrm{ad}}
\newcommand{\sch}{\mathrm{sch}}
\newcommand{\rig}{\mathrm{rig}}
\newcommand{\pol}{\mathrm{pol}}
\newcommand{\plat}{\mathrm{flat}}
\newcommand{\proper}{\mathrm{proper}}
\newcommand{\compl}{\mathrm{compl}}
\newcommand{\non}{\mathrm{non}}
\newcommand{\access}{\mathrm{access}}
\newcommand{\comp}{\mathrm{comp}}
\newcommand{\tstructure}{\mathrm{t}} %t-structures
\newcommand{\pure}{\mathrm{pure}} %pure motives
\newcommand{\mixed}{\mathrm{mixed}} %mixed motives
\newcommand{\num}{\mathrm{num}} %numerical motives
\newcommand{\ess}{\mathrm{ess}}
\newcommand{\topological}{\mathrm{top}}
\newcommand{\convex}{\mathrm{cvx}}
\newcommand{\locconvex}{\mathrm{lcvx}}
\newcommand{\ab}{\mathrm{ab}} %abelian extensions
\newcommand{\surj}{\mathrm{surj}} %coverage on sets generated by surjections
\newcommand{\eff}{\mathrm{eff}} %effective Cartier divisors
\newcommand{\Weil}{\mathrm{Weil}} %weil divisors
\newcommand{\lex}{\mathrm{lex}}
\newcommand{\rex}{\mathrm{rex}}
\newcommand{\AR}{\mathrm{A\-R}}
\newcommand{\cons}{\mathrm{c}} %constructible sheaves
\newcommand{\tor}{\mathrm{tor}} %tor dimension
\newcommand{\semisimple}{\mathrm{ss}}
\newcommand{\connected}{\mathrm{connected}}
\newcommand{\cg}{\mathrm{cg}} %compactly generated
\newcommand{\nilp}{\mathrm{nilp}}
\newcommand{\isg}{\mathrm{isg}} %isogenous
\newcommand{\qisg}{\mathrm{qisg}} %quasi-isogenous

%prism custom command
\usepackage{relsize}
\usepackage[bbgreekl]{mathbbol}
\usepackage{amsfonts}
\DeclareSymbolFontAlphabet{\mathbb}{AMSb} %to ensure that the meaning of \mathbb does not change
\DeclareSymbolFontAlphabet{\mathbbl}{bbold}
\newcommand{\prism}{{\mathlarger{\mathbbl{\Delta}}}}

\begin{document}

	\title{Modular varieties and the Langlands Correspondences}
	
	\author{Dat Minh Ha}
	\maketitle
	
	\begin{abstract}
	    
	\end{abstract}
	
	{
      \hypersetup{} 
      \dominitoc
      \tableofcontents %sort sections alphabetically
    }
    
    \part{Shimura varieties and the Global Langlands Correspondence over number fields}
        \chapter{Shimura varieties}
            \begin{abstract}
                
            \end{abstract}
            
            \minitoc
            
            \section{Shimura varieties over global number fields}
    \subsection{Locally symmetric spaces and connected Shimura varieties}
        \subsubsection{Symmetric spaces}
            \begin{definition}[Sequilinear forms] \label{def: sequilinear_forms}
                For any complex vector space $V$, a map $\<-,-\>: V \x V \to \bbC$ is said to be \textbf{sequilinear} if and only if for all vectors $v \in V$, the map $\<v, -\>: V \to \bbC$ is a $\bbC$-vector space homomorphism, while $\<-, v\>: V \to \bbC$ is an abelian group homomorphism that is $\bbC$-conjugate-linear, i.e. for all $\lambda \in \bbC$ and all $u \in V$, one has $\<\lambda u, v\> = \overline{\lambda} \<u, v\>$. Per the universal property of tensor products (of $\bbC$-vector spaces), a \textbf{sequilinear form} may also be viewed as an element $\<-,-\> \in (\overline{V} \tensor_{\bbC} V)^{\vee} := \Hom_{\bbC}(\overline{V} \tensor_{\bbC} V, \bbC)$, where $\overline{V}$ denotes the complex-conjugate of $V$.
            \end{definition}
            \begin{remark}[Hermitian-duals of sequilinear forms]
                Given any sequilinear form $\varphi \in (\overline{V} \tensor_{\bbC} V)^{\vee}$, there is a natural accompanying sequilinear form $\varphi^{\dagger} \in (V \tensor_{\bbC} \overline{V})^{\vee}$, called the \textbf{Hermitian dual} of $\varphi$. It is given by complex-conjugation followed by transposition (or vice versa), i.e. $\varphi^{\dagger}(u, v) := {}^t\overline{\varphi(v, u)}$.
            \end{remark}
            \begin{definition}[Hermitian forms] \label{def: hermitian_forms}
                A sequilinear form $\varphi$ on a $\bbC$-vector space $V$ is \textbf{Hermitian} if and only if it is Hermitian-self-dual, i.e. if and only if $\varphi^{\dagger} = \varphi$.
            \end{definition}
            \begin{definition}[Hermitian manifolds] \label{def: hermitian_manifolds}
                A \textbf{Hermitian manifold} is a Riemannian\footnote{This means that if $(M, J)$ is a complex manifold of complex dimension $d \geq 0$ then $(M, g)$ will be a real manifold of real dimension $2d$.} complex manifold $(M, J, g)$ such that $g$ is bi-invariant with respect to the complex structure $J$, i.e. $g(J(v), J(w)) = g(v, w)$ at every point $x \in M$ and all vector fields $v, w \in T_xM$.
            \end{definition}
            
            \begin{convention}[Automorphisms of manifolds]
                From now on, if $(M, J)$ is a complex manifold then its group of (biholomorphic) automorphisms\footnote{Note that biholomorphic maps are isomorphisms in the category of complex manifolds.} shall be denoted by $\Bi\Hol(M, J)$, and if $(M, g)$ is a (complex) Riemannian manifold then we shall write $\Isom(M, g)$ for its group of (holomorphic) isometries\footnote{Note that these are isomorphisms in the category of Riemannian manifolds.}.
            \end{convention}
            \begin{definition}[Homogeneous manifolds] \label{def: homogeneous_manifolds}
                A manifold $M$ is \textbf{homogeneous} if and only if it carries a transitive action of $\Aut(M)$. In particular, this means that a complex manifold $(M, J)$ (respectively, a Riemannian manifold $(M, g)$) is homogeneous if and only if the group $\Bi\Hol(M, J)$ (respectively, the group $(M, g)$) acts transitively on it.
            \end{definition}
            \begin{definition}[Symmetric spaces] \label{def: symmetric_spaces}
                A homogeneous manifold $M$ is said to be \textbf{symmetric} if and only if at each point $x \in M$, there is an involutive automorphism $\sigma_x \in \GL(T_xM)$ for which only $0 \in T_xM$ is a fixed point. A symmetric Riemannian (respectively, Hermitian) manifold that is also connected is known as a \textbf{symmetric space} (respectively, a \textbf{Hermitian symmetric space}).
            \end{definition}
            \begin{remark}[Symmetric spaces are globally symmetric]
                By homogeneity, the tangent space $T_xM$ at any point $x \in M$ of a symmetric manifold $M$ carries an involutive automorphism $\sigma_x \in \GL(T_xM)$ for which only $0 \in T_xM$ is a fixed point. In this sense, symmetric manifolds as in definition \ref{def: symmetric_spaces} are \textit{globally} symmetric.
            \end{remark}
            \begin{example}[The upper half-plane] \label{example: upper_half_plane_is_a_symmetric_space}
                Let $\h := \{z \in \bbC \mid \Im(z) > 0 \}$ denote the complex upper half-plane, which we shall regard as a complex manifold of complex dimension $1$ with respect to the obvious complex structure (check this!). It can also be endowed with a Riemannian metric (known commonly as the \textbf{Poincar\'e metric}) given by $\sqrt{g(z)} := \frac{dz d\overline{z}}{\Im(z)^2}$, making it a Riemannian manifold of real dimension $2$ (check this too!). By regarding $\h$ as a Riemannian of real dimension $2$, we may think of it as the following subset of $\Mat_2(\R)$:
                    $$
                        \h :=
                        \left\{
                            \begin{pmatrix}
                                x & -y
                                \\
                                y & x
                            \end{pmatrix}
                            \in \Mat_2(\R)
                            \:
                            \bigg|
                            \:
                            y > 0
                        \right\}
                    $$
                and so $\Isom(\h, g)$ is nothing but the subgroup of $\GL_2$ consisting of $2 \x 2$ invertible real matrices with determinant $1$ (so that they are isometries with respect to $g$), but this just means that $\Isom(\h, g) \cong \SL_2(\R)$. From here, one can show that the $\SL_2(\R)$-action given by:
                    $$\begin{pmatrix}a & b\\c & d\end{pmatrix} \mapsto \left(z \mapsto \frac{az + b}{cz + d}\right)$$
                is in fact transitive. $(\h, g)$ is thus a homogeneous complex Riemannian manifold. Now, in order to show that it is a symmetric space, observe that complex-conjugation is an involutive isometry at every point $z \in \h$, so the fact is trivial. Lastly, $\h$ is clearly connected, so it is indeed a symmetric space in the sense of definition \ref{def: symmetric_spaces}.
            \end{example}
            \begin{example}[Complex projective line] \label{example: complex_projective_line_is_a_symmetric_space}
                    
            \end{example}
            \begin{example}[Complex elliptic curves] \label{example: complex_elliptic_curves_are_symmetric_spaces}
                
            \end{example}
            Let us now single out a particular class of examples, that of so-called bounded symmetric domains, for further discussions. These entities will come up again later when we proceed onto the construction of Shimura varieties. 
            \begin{definition}[Domains] \label{def: complex_domains}
                A \textbf{complex domain} is a non-empty connected open subset of $\bbC^n$, for some $n \geq 0$. Such a complex domain is said to be bounded if and only if the underlying open subset of $\bbC^n$ is bounded. A complex domain, when regarded as a complex manifold with the obvious complex structure, is called \textbf{symmetric} if and only if it is so as a complex manifold. 
            \end{definition}
            \begin{proposition}[Bounded domains are Hermitian] \label{prop: bounded}
                Every bounded complex domain $U$ has a canonical $\Bi\Hol(U)$-invariant Hermitian metric, commonly called the \textbf{Bergmann metric}.
            \end{proposition}
                \begin{proof}
                    
                \end{proof}
            \begin{corollary}[Bounded domains are Hermitian symmetric spaces]
                Bounded complex domains when equipped with the Bergmann metric become Hermitian symmetric spaces. 
            \end{corollary}
                \begin{proof}
                    
                \end{proof}
            \begin{example}[]
                
            \end{example}
            
        \subsubsection{Classification of Hermitian symmetric spaces}
        
        \subsubsection{Locally symmetric spaces}
            
    \subsection{Construction of Shimura varieties}    
        \subsubsection{Congruence subgroups and connected Shimura varieties}
            \begin{definition}[The Deligne torus] \label{def: the_deligne_torus}
                The \textbf{Deligne torus}, which we shall denote by $\bbS_{\R}$, is the restriction of scalars\footnote{For this definition, recall that there is an adjoint equivalence (via the adjunction $\Gamma \ladjoint \Spec$) between the category of affine group $\bbC$-schemes and that of commutative Hopf $\bbC$-algebras, which we can apply the restriction of scalars functor $\Res^{\bbC}_{\R}$ to.} $\Res^{\bbC}_{\R} (\G_m)_{\bbC} := \Spec \Res^{\bbC}_{\R} \Gamma((\G_m)_{\bbC}, \calO_{(\G_m)_{\bbC}})$ of the multiplicative group $\bbC$-scheme $(\G_m)_{\bbC}$ down to $\R$. A \textbf{Deligne cocharacter} of a group $\R$-scheme $H$ is a morphism of group $\R$-schemes $h: \bbS_{\R} \to H^{\ad}$; the set of Deligne cocharacters of a given group $\R$-scheme $H$ is denoted by $\bbX^*(H)$.
            \end{definition}
            \begin{definition}[Connected Shimura data] \label{def: connected_shimura_data}
                A \textbf{connected Shimura datum} is a pair $(G, X^+)$ consisting of:
                    \begin{enumerate}
                        \item a connected semi-simple group $\Q$-scheme, and
                        \item a $G^{\ad}(\R)^+$-conjugacy class $X^+$ of Deligne cocharacters of $G^{\ad}_{\R}$ (i.e. $X^+$ is a subset of the $G^{\ad}(\R)^+$-fixed point set $\bbX^*(G^{\ad}_{\R})^{G^{\ad}(\R)^+}$ with respect to the conjugation action) satisfying the following axioms:
                            \begin{itemize}
                                \item For any $h \in X^+$ 
                                \item 
                            \end{itemize}
                    \end{enumerate}
            \end{definition}
            \begin{definition}[Principal congruence subgroups] \label{def: principal_congruence_subgroups}
                Let $G$ be a semi-simple linear algebraic group over $\Q$, and fix a closed immersion $G \subset (\GL_n)_{\Q}$ of $\Q$-varieties. For any positive integer $N$, define the \textbf{$N^{th}$ principal congruence subgroup} $\Gamma_G(N) \leq G(\Z)$ as the kernel of the canonically induced group homomorphism $G(\Z) \to G(\Z/N\Z)$. Explicitly, this means that:
                    $$\Gamma_G(N) := G(\Q) \cap \{g \in \GL_n(\Z) \mid g \equiv 1 \pmod{N}\}$$
            \end{definition}
    
        \subsubsection{Shimura varieties of Hodge type: moduli spaces of Hodge structures}
        
        \subsubsection{Shimura varieties of type PEL: moduli spaces of abelian varieties}
        
        \subsubsection{General Shimura varieties}
        
    \subsection{Complex multiplication}
            
            \section{Integral models of Shimura varieties}
            
            \section{Points modulo good reduction primes of Shimura varieties}
            
        \chapter{Modularity over global number fields}
            \begin{abstract}
                
            \end{abstract}
            
            \minitoc
            
        \begin{appendices}
            \chapter{The arithmetic of hyperbolic \texorpdfstring{$3$}{}-manifolds}
                \begin{abstract}
                    
                \end{abstract}
                
                \minitoc
                
                \section{Kleinian groups}
                
                \section{Invariant trace fields}
                
                \section{Quaternion algebras}
                
                \section{Arithmetic hyperbolic \texorpdfstring{$3$}{}-manifolds}
                
            \chapter{Abelian varieties}
                \begin{abstract}
                    
                \end{abstract}
                
                \minitoc
                
                \section{The geometry of abelian varieties}
    \subsection{Abelian schemes}
        \subsubsection{Rigidity and commutative group structures on abelian schemes}
            \begin{definition}[Abelian schemes] \label{def: abelian_schemes}
                Let $S$ be a base scheme. An \textbf{abelian $S$-scheme} is defined via a proper and smooth structural morphism $\pi: A \to S$ whose fibres over Zariski-points $s \in |S|$ are geometrically connected. 
            \end{definition}
            \begin{proposition}[Abelian schemes over fields are varieties] \label{prop: abelian_schemes_over_fields_are_varieties}
                Abelian schemes over fields are algebraic varieties (and as such are called \textbf{abelian varieties}).
            \end{proposition}
                \begin{proof}
                        
                \end{proof}
            \begin{corollary}[Abelian varieties are geometrically integral] \label{coro: abelian_varieties_are_geometrically_integral}
                Abelian varieties are geometrically integral.
            \end{corollary}
                \begin{proof}
                    
                \end{proof}
            \begin{example}
                Elliptic curves are abelian varieties of dimension $1$.
            \end{example}
            \begin{example}
                Due to the assumption of having geometrically connected fibres over Zariski-points, there are no $0$-dimensional abelian varieties (as spectra of field extensions are totally disconnected in the Zariski topology).
            \end{example}
            \begin{proposition}
                Let $f: T \to S$ be any morphism of schemes and let $\pi: A \to S$ be an abelian scheme over $S$. Then the pullback $A_T := A \x_{\pi, S, f} T$ will be an abelian scheme over $T$.
            \end{proposition}
                \begin{proof}
                    
                \end{proof}
                
            As a test case, let us establish the following fact: that complex abelian varieties come naturally equipped with commutative group $\bbC$-variety structures. Generalising this fact to the case of abelian schemes over an arbitrary base will not be a trivial task, however, because for complex abelian varieties, the proof involves using the exponential map - which is transcendental -3 to globalise commutative Lie algebra structures on tangent spaces to the whole variety. 
            \begin{proposition}[Uniformisations of complex abelian varieties] \label{prop: uniformisations_of_complex_abelian_varieties}
                Let $A$ be an abelian $\bbC$-variety. Then the associated complex manifold $A^{\an} := A(\bbC)$ is biholomorphic to $\bbC^g/\Lambda$ for some\footnote{Later on, we will see that $g$ is actually the genus of $A$.} $g \geq 1$ and some full-rank free $\Z$-submodule $\Lambda \subset \bbC^g$.
            \end{proposition}
                \begin{proof}
                    
                \end{proof}
            \begin{corollary}[Complex abelian varieties are Lie groups] \label{coro: complex_abelian_varieties_are_complex_lie_groups}
                Any given complex abelian variety $A$ carries a natural commutative group $\bbC$-variety structure. In fact, $A^{\an}$ is a compact and connected complex Lie group.
            \end{corollary}
            \begin{example}
                Because elliptic curves are abelian varieties of dimension $1$, complex elliptic curves uniformise to quotient complex manifolds of the form $\bbC/\Lambda$, where $\Lambda$ is a non-zero cyclic group.
            \end{example}
            \begin{remark}[Torsion subgroups of complex abelian varieties]
                Proposition \ref{prop: uniformisations_of_complex_abelian_varieties}, in addition to serving as a test case for the establishment of the much more general fact that abelian schemes are commutative group schemes, also hints at how one might consider torsion subgroups of abelian varieties. In particular, for some complex abelian variety $A$ and uniformisation $\bbC^g/\Lambda$, one may define the $n$-torsion subgroup of the associated manifold $A^{\an}$ as:
                    $$A^{\an}[n] := A(\bbC)[n]$$
                and then see that:
                    $$A^{\an}[n] \cong \Tor_1^{\Z}(\Z/n\Z, \bbC^g/\Lambda) \cong (\Z/n\Z)^{\oplus 2g}$$
                As for applicatinos, notice that by letting $n$ vary through the powers of a prime $\ell$ when $g = 1$ (i.e. the case of complex elliptic curves), and by supposing furthermore that the complex elliptic curve $E_{\bbC}$ in question is actually the fibre over $\bbC$ of some elliptic curve over a number field $K/\Q$, this realisation of torsion subgroups $E_{\bbC}^{\an}[n]$ allows us to construct $2$-dimensional Galois representations:
                    $$\rho_{E_{\bbC}}[\ell^r]: \Gal(\bar{K}/K) \to \GL_2(\Z/\ell^r\Z)$$
                Taking the limit and tensoring over $\Z_{\ell}$ with $\bar{\Q}_{\ell}$ then yields us $2$-dimensional $\ell$-adic Galois representations:
                    $$\rho_{E_{\bbC}, \bar{\Q}_{\ell}}: \Gal(\bar{K}/K) \to \GL_2(\bar{\Q}_{\ell})$$
            \end{remark}
            
            Now, in order to generalise proposition \ref{prop: uniformisations_of_complex_abelian_varieties} to cases wherein the underlying base scheme is arbitrary and not just $\Spec \bbC$, we shall need to prove the so-called \textbf{Rigidity Theorems}, which are scheme-theoretic analogues of the fact that bounded and globally holomorphic functions are necessarily constant.
            \begin{theorem}[Rigidity I] \label{theorem: rigidity_theorem_for_abelian_varieties_1}
                Let $\pi: X \to S$ be a proper and flat morphism of schemes such that\footnote{I.e. the morphism $\pi: X \to S$ is pointwise constant.} $\kappa_s \cong H^0_{\Zar}(X_s, \calO_{X_s})$ for all $s \in |S|$. Then the canonically defined $\calO_S$-linear map $\pi^{\sharp}: \calO_S \to \pi_*\calO_X$ will be an isomorphism.
            \end{theorem}
                \begin{proof}
                    
                \end{proof}
            \begin{corollary}[Liouville's theorem on entire functions for schemes] \label{coro: liouville_theorem_on_entire_functions_for_schemes}
                Let $\pi: X \to S$ be a proper and flat morphism of schemes such that $\kappa_s \cong H^0(X_s, \calO_{X_s})$ for all $s \in |S|$. For all affine morphisms $f: T \to S$ for which there exists a factorisation:
                    $$
                        \begin{tikzcd}
                        	& X \\
                        	T & S
                        	\arrow["\pi", from=1-2, to=2-2]
                        	\arrow["f", from=2-1, to=2-2]
                        	\arrow["\varphi"', dashed, from=1-2, to=2-1]
                        \end{tikzcd}
                    $$
                the morphism of $S$-schemes $\varphi: X \to T$ is necessarily constant.
            \end{corollary}
                \begin{proof}
                    
                \end{proof}
            \begin{remark}
                When $S$ is the spectrum of a field, the assumption that $\kappa_s \cong H^0(X_s, \calO_{X_s})$ for all $s \in |S|$ is equivalent to the assumption that the morphism $f_s: X_s \to \Spec \kappa_s$ is geometrically integral at all $s \in |S|$.
            \end{remark}
            A main ingredient for the second Rigidity Theorem is Zariski's Main Theorem, which asserts that quasi-finite and separated morphisms over a qcqs base factors into a quasi-compact open immersion followed by a finite morphism.
            \begin{lemma}[Zariski's Main Theorem] \label{lemma: zasriski_main_theorem}
                \cite[\href{https://stacks.math.columbia.edu/tag/05K0}{Tag 05K0}]{stacks} Let $S$ be a qcqs scheme and let $f: T \to S$ be a quasi-finite and separated morphism. This morphism factors into a quasi-compact open immersion $j: T \to X$ followed by a finite morphism $\pi: X \to S$ as follows:
                    $$
                        \begin{tikzcd}
                        	& X \\
                        	T & S
                        	\arrow["f"', from=2-1, to=2-2]
                        	\arrow["j", hook, from=2-1, to=1-2]
                        	\arrow["\pi", from=1-2, to=2-2]
                        \end{tikzcd}
                    $$
            \end{lemma}
                \begin{proof}
                    
                \end{proof}
            \begin{proposition}[Finite = proper + finite fibres] \label{lemma: finite_iff_proper_with_finite_fibres}
                
            \end{proposition}
                \begin{proof}
                    
                \end{proof}
            \begin{corollary}
                
            \end{corollary}
                \begin{proof}
                    
                \end{proof}
            \begin{theorem}[Rigidity II] \label{theorem: rigidity_theorem_for_abelian_varieties_2}
                \cite[\href{https://stacks.math.columbia.edu/tag/0AH8}{Tag 0AH8}]{stacks} Consider a commutative diagram of schemes:
                    $$
                        \begin{tikzcd}
                        	& X \\
                        	T & S
                        	\arrow["\pi", from=1-2, to=2-2]
                        	\arrow["f", from=2-1, to=2-2]
                        	\arrow["\varphi"', from=1-2, to=2-1]
                        \end{tikzcd}
                    $$
                wherein $\pi: X \to S$ is proper, $f: T \to S$ is separated and locally of finite type, and the scheme-theoretic image of $\varphi_s: X_s \to T_s$ is finite over $S$. In such a situation, there exists a Zariski-open neighbourhood $U \ni s$ such that there exists a factorisation as follows:
                    $$
                        \begin{tikzcd}
                        	Z & {X_U} \\
                        	{T_U}
                        	\arrow["\varphi", from=1-2, to=2-1]
                        	\arrow[hook, from=1-1, to=2-1]
                        	\arrow[dashed, from=1-2, to=1-1]
                        \end{tikzcd}
                    $$
                of $\varphi_U: X_U \to T_U$ via a closed subscheme $Z \subseteq T_U$ that is finite over $U$.
            \end{theorem}
                \begin{proof}
                    
                \end{proof}
            \begin{corollary}[Abelian schemes are commutative group schemes] \label{coro: abelian_schemes_are_commutative_group_schemes} 
                Let $A$ be an abelian $S$-scheme. Then $A$ carries a \textit{unqiue} commutative group $S$-scheme structure.
            \end{corollary}
                \begin{proof}
                    
                \end{proof}
                
        \subsubsection{Tangent spaces}
        
        \subsubsection{Cohomology of abelian varieties}
        
    \subsection{Line bundles on abelian varieties}
        \subsubsection{The Seesaw Principle}
        
        \subsubsection{The Theorem of the Cube}
        
        \subsubsection{Projectivity of abelian varieties}
                
                \section{Dual abelian varieties}
                
                \section{Categorical harmonic analysis on abelian varieties}
                
            \chapter{The Tannakian formalism}
                \begin{abstract}
                    
                \end{abstract}
                
                \minitoc
                
                \section{Tannakian categories}
                
                \section{Motives for absolute Hodge cycles}
        \end{appendices}
    
    \part{Drinfeld modular varieties and the Global Langlands Correspondence over finite fields}
        \chapter{Drinfeld modular varieties}
            \begin{abstract}
                
            \end{abstract}
            
            \minitoc
            
            We subscribe to the following general conventions throughout the entire chapter:
            \begin{convention}
                Throughout the chapter, we work over a fixed smooth, projective, and geometrically connected curve $X$ over a finite field $k := \F_q$. Such a curve, by virtue of being an integral scheme, admits a unique generic point that we shall denote by $\eta$; the residue field of said generic point shall be denoted by $F$. The completion of $F$ at each closed point $x \in |X|$ shall be denoted by $F_x$, each of which is a non-archimedean discretely valued local field whose ring of intergers and residue field shall, respectively, be denoted by $\scrO_x$ and $\kappa_x$; we will also often pick and fix pseudo-uniformisers $\varpi_x \in \scrO_x$. In addition, the degree $\deg(x)$ of a given closed point $x \in |X|$ shall be the degree of the extension $[\kappa_x : \F_q]$. 
            \end{convention}
            
            \section{Construction of Drinfeld modular varieties}
            
            \section{Drinfeld modules}
            
            \section{Counting points on Drinfeld modular varieties}
            
        \chapter{Automorphic forms over global function fields}
        
    \part{The \texorpdfstring{$p$}{}-adic Local Langlands Correspondence via \texorpdfstring{$p$}{}-adic modular varieties}
        \chapter{The Langlands-Kottwitz Method}
            \begin{abstract}
                    
            \end{abstract}
            
            \minitoc
            
            \chapter{The Langlands-Kottwitz method for modular curves}
    \begin{abstract}
        We follow hereinafter a paper of Peter Scholze (cf. \cite{scholze_langlands_kottwitz_for_modular_curves}), wherein he explained how the Langland-Kottwitz Method for counting points of Shimura varieties could be used to determine local factors of Hasse-Weil $\zeta$-functions of modular curves at places of bad reduction as well as to prove a trace formula for $\ell$-adic cohomologies of Shimura varieties conjectured Haines and Kottwitz.
    \end{abstract}
    
    \minitoc
    
    \section{Introduction}
        Fix an integer $m$ and let $\pi: \calM_m \to \Spec \Z\left[\frac1m\right]$ be the moduli space of elliptic curves with level-$m$ structures and let $\pi_n: \calM_{\Gamma(p^n), m} \to \calM_m$ be the finite \'etale covering thereof by the moduli space of elliptic curves with Drinfeld level-$p^n$ structures and level-$m$ structures. By inverting the prime $p$ (i.e. by restricting down to a Zariski-open neighbourhood of the point $(p) \in \Spec \Z\left[\frac1m\right]$), one obtains a Galois covering $\pi_n\left[\frac1p\right]: \calM_{\Gamma(p^n), m}\left[\frac1p\right] \to \calM_m\left[\frac1p\right]$ whose Galois group is $\GL_2(\Z/p^n)$.
        
        Next, recall via the Proper Base Change Theorem for \'Etale Cohomology and the Lefschetz Trace Formula (also within the context of \'etale cohomology of algebraic varieties) that for $X$ a proper smooth variety over $\Spec \Q$ with good reduction at some given prime $p$, the local factor at $p$ of its Hasse-Weil $\zeta$-function is given by:
            $$\zeta_{X/p}(s) := \exp \sum_{r \geq 1} |\frakX(\F_{p^r})| \frac{p^{-rs}}{r}$$
        wherein $\frakX$ is some proper smooth model over $\Spec \Z_{(p)}$ of $X$, and the semi-simple factor is given by:
            $$\zeta_p^{\semisimple}(s) := \exp\left( \sum_{r \geq 1} \sum_{x \in |\calM_m(\F_{p^r})|} \trace^{\semisimple}(\Frob_{p^r} \mid ( \Psi^{\bullet} \bar{\Q}_{\ell})_x) \right) \frac{p^{-rs}}{r}$$
        wherein $\Psi^{\bullet}: D_{\lisse}^b(\frakX_{\eta}) \to D_{\lisse}^b(\frakX_{\F_p} \x_{\F_p} \eta)$ is the (derived) functor of nearby cycles on the formal scheme $\frakX$ over the Henselian trait $(\Spec \Z_{(p)}, \Spec \F_p, \eta := \Spec \Frac \Z_{(p)})$. 
    
    \section{The Langlands-Kottwitz for counting points of Shimura varieties}
            
            \chapter{The Langlands-Kottwitz method for Shimura varieties}
    \begin{abstract}
            
    \end{abstract}
    
    \minitoc
            
        \chapter{The Local Langlands Correspondence for \texorpdfstring{$p$}{}-adic \texorpdfstring{$\GL_n$}{}}
            \begin{abstract}
                    
            \end{abstract}
            
            \minitoc
        
    \begin{appendices}
        \chapter{Nearby and vanishing cycles}
            \begin{abstract}
                
            \end{abstract}
            
            \minitoc
            
            \section{Nearby and vanishing cycles on schemes}
            
            \section{Nearby and vanishing cycles on formal schemes}
    \end{appendices}
	
	\addcontentsline{toc}{section}{References}
	\printbibliography

\end{document}