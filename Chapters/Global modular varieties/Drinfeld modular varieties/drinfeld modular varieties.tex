\chapter{Drinfeld modular varieties}
    \begin{abstract}
        
    \end{abstract}
    
    We subscribe to the following general conventions throughout the entire chapter:
    \begin{convention}
        Throughout the chapter, we work over a fixed smooth, projective, and geometrically connected curve $X$ over a finite field $k := \F_q$. Such a curve, by virtue of being an integral scheme, admits a unique generic point that we shall denote by $\eta$; the residue field of said generic point shall be denoted by $F$. The completion of $F$ at each closed point $x \in |X|$ shall be denoted by $F_x$, each of which is a non-archimedean discretely valued local field whose ring of intergers and residue field shall, respectively, be denoted by $\scrO_x$ and $\kappa_x$; we will also often pick and fix pseudo-uniformisers $\varpi_x \in \scrO_x$. In addition, the degree $\deg(x)$ of a given closed point $x \in |X|$ shall be the degree of the extension $[\kappa_x : \F_q]$. 
    \end{convention}
    
    Let us also recall the following rather well-known result from basic algebraic geometry, which allows us to fix once and for all a closed point $x_{\infty} \in |X|$ for which $X \setminus \{x_{\infty}\}$ is affine (in fact, isomorphic to $\Spec A$)\footnote{$A$ for \say{a\`delic}.}:
    \begin{proposition}
        Suppose that $X$ is a smooth and geometrically connected smooth curve over a field $k$, and suppose that $x_1, ..., x_n \in |X|$ are finitely many arbitrarily chosen closed points of $X$. Then, the scheme $X \setminus \{x_1, ..., x_n\}$ will be affine.  
    \end{proposition}
    
    \section{Introduction}
        To any quadruple $(F, x_{\infty}, \GL_d, \frakN)$ consisting of the function field $F := k(X)$ of a suitable algebraic curve (smooth, projective, geometrically connected) over a finite field $k := \F_q$, a place $x_{\infty}$ of $F$, a general linear group scheme $\GL_d$ over $\Spec k$, and a non-zero ideal $\frakN \subset A$, one can associate an affine and smooth scheme of dimension $d - 1$ over $\Spec A \setminus V(\frakN)$, a so-called Drinfeld modular variety $\calM_{\GL_d, \frakN}$. Such a modular variety is a positive-characteristic analogue of Shimura varieties and as such parametrises \say{elliptic data}: just as how the modular curve $X(N)$ parametrises elliptic curves with level-$N$ structures, or how PEL-type Shimura varieties parametrise certain kinds of higher-dimensional abelian varieties, Drinfeld modular varieties parametrise so-called \say{elliptic modules} of rank $d$ and level-$\frakN$ structures. One can therefore employ them to construct and prove a version of the Langlands Correspondence for global function fields over finite fields, in the same manner that the modular curve was used for attempts at understanding the Modularity Theorem: the hope is that the richness of positive-characteristic geometry will shine some light on the mysteries of the Langlands Correspondence, thereby serving as a template for the Global Correspondence over number fields. 
        
        Now, by varying the given $A$-ideal $\frakN$, one obtains a cofiltered diagram of quasi-projective schemes $\{\calM_{\GL_d, \frakN}\}_{\text{$A$-ideal $\frakN$}}$ over $\Spec A \setminus V(\frakN)$ indexed by ideal containments, whose limit is an affine scheme $\calM_{\GL_d}$ over $\Spec F$ per some general algebraic geometry. An interesting feature of this new affine scheme $\calM_{\GL_d}$ is that it admits an action of the ad\`elic group $\GL_d(\A_X)$ (with $\A_X := F \tensor_A \hat{A}$ being the ring of rational a\`eles of the global function field of $X$). 

    \section{Construction of Drinfeld modular varieties}
        \subsection{Endomorphisms of commutative group schemes and Drinfeld modules}
            \begin{convention}[The additive group]
                From now on, $\G_a$ shall denote the additive (affine) group scheme over $\Spec k$. Recall that it is represented by the affine scheme $\Spec k[t]$.
            \end{convention}
            \begin{remark}[Endomorphisms of the additive group]
                The first observation that (for all commutative rings $k$ of characteristic $p > 0$) one could make is that $\End(\G_a/k)$ is a noncommutative ring which is isomorphic to the noncommutative polynomial ring $k\<\tau\>/\<\forall \alpha \in k: \alpha^p \tau = \tau \alpha\>$. 
            \end{remark}
    
            \begin{definition}[Drinfeld modules] \label{def: drinfeld_modules}
                Suppose that $S$ is a $k$-scheme and that $E/S$ is a commutative group scheme which is:
                    \begin{itemize}
                        \item Zariski-locally to $(\G_a)_S$ over $S$ (i.e. given any Zariski-open subscheme $U \hookrightarrow S$, the pullback $E \x_S U$ is isomorphic to $(\G_a \x_{\Spec k} S) \x_S U$) and
                        \item such that for each Zariski-local isomorphism $\psi_U: E_U \to (\G_a)_U$ (viewed as an automorphism $\psi_U \in \Aut(E_U/U)$), the conjugation $\psi_U \varphi_U \psi_U^{-1}$
                    \end{itemize}
                One can then equip the commutative group scheme $E$ with the structure of a \textbf{Drinfeld module} by letting $A$ acting on it via $\varphi: A \to \End(E/S)$. We shall often denote Drinfeld modules as pairs $(E, \varphi)$.
            \end{definition}
    
    \section{Drinfeld modules}
    
    \section{Counting points on Drinfeld modular varieties}