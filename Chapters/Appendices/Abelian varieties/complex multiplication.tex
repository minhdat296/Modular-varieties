\section{Complex multiplication}
    \subsection{Generalities}
        \subsubsection{Brauer groups}
    
        \subsubsection{Simplicity and isotypicity of abelian varieties}
            \begin{definition}[Simple abelian varieties] \label{def: simple_abelian_varieties}
                An abelian scheme $\pi: A \to S$ over some arbitrary base scheme $S$ is said to be \textbf{simple} if and only if its only abelian subschemes are the trivial abelian scheme $S$ and $A$ itself. When $S \cong \Spec K$ for some field $K$, one says that $A$ is \textbf{absolutely simple} (respectively, \textbf{geometrically simple}) if and only if $A_{\bar{K}}$ (respectively, $A_{K'}$ where $K'/K$ is any field extension) is simple as an abelian $\bar{K}$-variety.
            \end{definition}
            \begin{proposition}[Absolutely simple iff geometrically simple] \label{prop: absolutely_simple_iff_geometrically_simple}
                Abelian varieties are absolutely simply if and only if they are geometrically simple.
            \end{proposition}
                \begin{proof}
                    
                \end{proof}
        
        \subsubsection{Complex multiplication}