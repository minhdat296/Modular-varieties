\section{Nearby and vanishing cycles on schemes}
    We begin by fixing certain terminologies:
    \begin{convention}
        \noindent
        \begin{itemize}
            \item \textbf{(Henselian traits):} A \textbf{Henselian trait} (or simply \say{\textbf{trait}}) shall be a triple $(S, s, \eta)$ consisting of the affine scheme $S := \Spec R$ corresponding to a local Henselian ring $(R, \m)$ (e.g. discrete valuation rings) along with its special point $s$ (corresponding to the residue field $\kappa := R/\m$) and generic point $\eta$ (corresponding to the field of fractions $K := \Frac R$). A given trait $(S, s, \eta)$ is said to be \textbf{strict} if and only if its residue field $\kappa$ is separably closed. 
            \item \textbf{(Morphisms essentially of finite type):} 
        \end{itemize}
    \end{convention}
    
    \subsection{Introduction}
        
    \subsection{Vanishing cycles}