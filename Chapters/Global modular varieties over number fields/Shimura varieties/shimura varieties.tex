\section{Shimura varieties over global number fields}
    \subsection{Locally symmetric spaces and connected Shimura varieties}
        \subsubsection{Symmetric spaces}
            \begin{definition}[Sequilinear forms] \label{def: sequilinear_forms}
                For any complex vector space $V$, a map $\<-,-\>: V \x V \to \bbC$ is said to be \textbf{sequilinear} if and only if for all vectors $v \in V$, the map $\<v, -\>: V \to \bbC$ is a $\bbC$-vector space homomorphism, while $\<-, v\>: V \to \bbC$ is an abelian group homomorphism that is $\bbC$-conjugate-linear, i.e. for all $\lambda \in \bbC$ and all $u \in V$, one has $\<\lambda u, v\> = \overline{\lambda} \<u, v\>$. Per the universal property of tensor products (of $\bbC$-vector spaces), a \textbf{sequilinear form} may also be viewed as an element $\<-,-\> \in (\overline{V} \tensor_{\bbC} V)^{\vee} := \Hom_{\bbC}(\overline{V} \tensor_{\bbC} V, \bbC)$, where $\overline{V}$ denotes the complex-conjugate of $V$.
            \end{definition}
            \begin{remark}[Hermitian-duals of sequilinear forms]
                Given any sequilinear form $\varphi \in (\overline{V} \tensor_{\bbC} V)^{\vee}$, there is a natural accompanying sequilinear form $\varphi^{\dagger} \in (V \tensor_{\bbC} \overline{V})^{\vee}$, called the \textbf{Hermitian dual} of $\varphi$. It is given by complex-conjugation followed by transposition (or vice versa), i.e. $\varphi^{\dagger}(u, v) := {}^t\overline{\varphi(v, u)}$.
            \end{remark}
            \begin{definition}[Hermitian forms] \label{def: hermitian_forms}
                A sequilinear form $\varphi$ on a $\bbC$-vector space $V$ is \textbf{Hermitian} if and only if it is Hermitian-self-dual, i.e. if and only if $\varphi^{\dagger} = \varphi$.
            \end{definition}
            \begin{definition}[Hermitian manifolds] \label{def: hermitian_manifolds}
                A \textbf{Hermitian manifold} is a Riemannian\footnote{This means that if $(M, J)$ is a complex manifold of complex dimension $d \geq 0$ then $(M, g)$ will be a real manifold of real dimension $2d$.} complex manifold $(M, J, g)$ such that $g$ is bi-invariant with respect to the complex structure $J$, i.e. $g(J(v), J(w)) = g(v, w)$ at every point $x \in M$ and all vector fields $v, w \in T_xM$.
            \end{definition}
            
            \begin{convention}[Automorphisms of manifolds]
                From now on, if $(M, J)$ is a complex manifold then its group of (biholomorphic) automorphisms\footnote{Note that biholomorphic maps are isomorphisms in the category of complex manifolds.} shall be denoted by $\Bi\Hol(M, J)$, and if $(M, g)$ is a (complex) Riemannian manifold then we shall write $\Isom(M, g)$ for its group of (holomorphic) isometries\footnote{Note that these are isomorphisms in the category of Riemannian manifolds.}.
            \end{convention}
            \begin{definition}[Homogeneous manifolds] \label{def: homogeneous_manifolds}
                A manifold $M$ is \textbf{homogeneous} if and only if it carries a transitive action of $\Aut(M)$. In particular, this means that a complex manifold $(M, J)$ (respectively, a Riemannian manifold $(M, g)$) is homogeneous if and only if the group $\Bi\Hol(M, J)$ (respectively, the group $(M, g)$) acts transitively on it.
            \end{definition}
            \begin{definition}[Symmetric spaces] \label{def: symmetric_spaces}
                A homogeneous manifold $M$ is said to be \textbf{symmetric} if and only if at each point $x \in M$, there is an involutive automorphism $\sigma_x \in \GL(T_xM)$ for which only $0 \in T_xM$ is a fixed point. A symmetric Riemannian (respectively, Hermitian) manifold that is also connected is known as a \textbf{symmetric space} (respectively, a \textbf{Hermitian symmetric space}).
            \end{definition}
            \begin{remark}[Symmetric spaces are globally symmetric]
                By homogeneity, the tangent space $T_xM$ at any point $x \in M$ of a symmetric manifold $M$ carries an involutive automorphism $\sigma_x \in \GL(T_xM)$ for which only $0 \in T_xM$ is a fixed point. In this sense, symmetric manifolds as in definition \ref{def: symmetric_spaces} are \textit{globally} symmetric.
            \end{remark}
            \begin{example}[The upper half-plane] \label{example: upper_half_plane_is_a_symmetric_space}
                Let $\h := \{z \in \bbC \mid \Im(z) > 0 \}$ denote the complex upper half-plane, which we shall regard as a complex manifold of complex dimension $1$ with respect to the obvious complex structure (check this!). It can also be endowed with a Riemannian metric (known commonly as the \textbf{Poincar\'e metric}) given by $\sqrt{g(z)} := \frac{dz d\overline{z}}{\Im(z)^2}$, making it a Riemannian manifold of real dimension $2$ (check this too!). By regarding $\h$ as a Riemannian of real dimension $2$, we may think of it as the following subset of $\Mat_2(\R)$:
                    $$
                        \h :=
                        \left\{
                            \begin{pmatrix}
                                x & -y
                                \\
                                y & x
                            \end{pmatrix}
                            \in \Mat_2(\R)
                            \:
                            \bigg|
                            \:
                            y > 0
                        \right\}
                    $$
                and so $\Isom(\h, g)$ is nothing but the subgroup of $\GL_2$ consisting of $2 \x 2$ invertible real matrices with determinant $1$ (so that they are isometries with respect to $g$), but this just means that $\Isom(\h, g) \cong \SL_2(\R)$. From here, one can show that the $\SL_2(\R)$-action given by:
                    $$\begin{pmatrix}a & b\\c & d\end{pmatrix} \mapsto \left(z \mapsto \frac{az + b}{cz + d}\right)$$
                is in fact transitive. $(\h, g)$ is thus a homogeneous complex Riemannian manifold. Now, in order to show that it is a symmetric space, observe that complex-conjugation is an involutive isometry at every point $z \in \h$, so the fact is trivial. Lastly, $\h$ is clearly connected, so it is indeed a symmetric space in the sense of definition \ref{def: symmetric_spaces}.
            \end{example}
            \begin{example}[Complex projective line] \label{example: complex_projective_line_is_a_symmetric_space}
                    
            \end{example}
            \begin{example}[Complex elliptic curves] \label{example: complex_elliptic_curves_are_symmetric_spaces}
                
            \end{example}
            Let us now single out a particular class of examples, that of so-called bounded symmetric domains, for further discussions. These entities will come up again later when we proceed onto the construction of Shimura varieties. 
            \begin{definition}[Domains] \label{def: complex_domains}
                A \textbf{complex domain} is a non-empty connected open subset of $\bbC^n$, for some $n \geq 0$. Such a complex domain is said to be bounded if and only if the underlying open subset of $\bbC^n$ is bounded. A complex domain, when regarded as a complex manifold with the obvious complex structure, is called \textbf{symmetric} if and only if it is so as a complex manifold. 
            \end{definition}
            \begin{proposition}[Bounded domains are Hermitian] \label{prop: bounded}
                Every bounded complex domain $U$ has a canonical $\Bi\Hol(U)$-invariant Hermitian metric, commonly called the \textbf{Bergmann metric}.
            \end{proposition}
                \begin{proof}
                    
                \end{proof}
            \begin{corollary}[Bounded domains are Hermitian symmetric spaces]
                Bounded complex domains when equipped with the Bergmann metric become Hermitian symmetric spaces. 
            \end{corollary}
                \begin{proof}
                    
                \end{proof}
            \begin{example}[]
                
            \end{example}
            
        \subsubsection{Classification of Hermitian symmetric spaces}
        
        \subsubsection{Locally symmetric spaces}
            
    \subsection{Construction of Shimura varieties}    
        \subsubsection{Congruence subgroups and connected Shimura varieties}
            \begin{definition}[The Deligne torus] \label{def: the_deligne_torus}
                The \textbf{Deligne torus}, which we shall denote by $\bbS_{\R}$, is the restriction of scalars\footnote{For this definition, recall that there is an adjoint equivalence (via the adjunction $\Gamma \ladjoint \Spec$) between the category of affine group $\bbC$-schemes and that of commutative Hopf $\bbC$-algebras, which we can apply the restriction of scalars functor $\Res^{\bbC}_{\R}$ to.} $\Res^{\bbC}_{\R} (\G_m)_{\bbC} := \Spec \Res^{\bbC}_{\R} \Gamma((\G_m)_{\bbC}, \calO_{(\G_m)_{\bbC}})$ of the multiplicative group $\bbC$-scheme $(\G_m)_{\bbC}$ down to $\R$. A \textbf{Deligne cocharacter} of a group $\R$-scheme $H$ is a morphism of group $\R$-schemes $h: \bbS_{\R} \to H^{\ad}$; the set of Deligne cocharacters of a given group $\R$-scheme $H$ is denoted by $\bbX^*(H)$.
            \end{definition}
            \begin{definition}[Connected Shimura data] \label{def: connected_shimura_data}
                A \textbf{connected Shimura datum} is a pair $(G, X^+)$ consisting of:
                    \begin{enumerate}
                        \item a connected semi-simple group $\Q$-scheme, and
                        \item a $G^{\ad}(\R)^+$-conjugacy class $X^+$ of Deligne cocharacters of $G^{\ad}_{\R}$ (i.e. $X^+$ is a subset of the $G^{\ad}(\R)^+$-fixed point set $\bbX^*(G^{\ad}_{\R})^{G^{\ad}(\R)^+}$ with respect to the conjugation action) satisfying the following axioms:
                            \begin{itemize}
                                \item For any $h \in X^+$ 
                                \item 
                            \end{itemize}
                    \end{enumerate}
            \end{definition}
            \begin{definition}[Principal congruence subgroups] \label{def: principal_congruence_subgroups}
                Let $G$ be a semi-simple linear algebraic group over $\Q$, and fix a closed immersion $G \subset (\GL_n)_{\Q}$ of $\Q$-varieties. For any positive integer $N$, define the \textbf{$N^{th}$ principal congruence subgroup} $\Gamma_G(N) \leq G(\Z)$ as the kernel of the canonically induced group homomorphism $G(\Z) \to G(\Z/N\Z)$. Explicitly, this means that:
                    $$\Gamma_G(N) := G(\Q) \cap \{g \in \GL_n(\Z) \mid g \equiv 1 \pmod{N}\}$$
            \end{definition}
    
        \subsubsection{Shimura varieties of Hodge type: moduli spaces of Hodge structures}
        
        \subsubsection{Shimura varieties of type PEL: moduli spaces of abelian varieties}
        
        \subsubsection{General Shimura varieties}
        
    \subsection{Complex multiplication}